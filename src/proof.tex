\section{Proof of the main theorem}

\begin{frame}
  \centering
  \hfill\\
  \begin{theorem}
    Let $C$ be a Clifford circuit measuring computing Pauli operators $S_1, \ldots, S_r$.
    Then, for any subset of qubits $L$, we have
    \begin{equation*}
      \text{depth}(C) 
      \geq
      \frac
        {n_{\text{cut}}}
        {64 |\partial L|}.
    \end{equation*}
  \end{theorem}
  \begin{corollary}
    For families of local-expander quantum LDPC codes of length $n$,
    a syndrome extraction circuit $C$ implemented as a local Clifford circuit 
    on a $\sqrt{N} \times \sqrt{N}$ grid of qubits
    satisfies
    \begin{equation*}
      \text{depth}(C) 
      \geq
      \Omega \left(
        \frac 
          {n}
          {\sqrt{N}}
      \right).
    \end{equation*}
  \end{corollary}
\end{frame}

\begin{frame}{Strategy}
  \hfill\\
  \begin{itemize}
    \item Partition the circuit's qubits into two subsets $L$ and $R$.
    \pause
    \item Lower bound the amount of correlation required between $L$ and $R$ 
    to measure the Pauli operators.
    \pause
    \item Upper bound the amount of correlation introduced per operation.
    \pause
    \item Combine both arguments to derive a lower bound for the depth of the circuit.
  \end{itemize}
\end{frame}

\begin{frame}{Measuring correlations}
  \hfill\\
  \Large
  \begin{columns}[c]
    \begin{column}{0.4\textwidth}
      \centering
      \begin{equation*}
        \Qcircuit @C=1em @R=1em {
          \lstick{\ket{0}} & \multigate{4}{C_{XX}} & \qw & \qw      & \qw & \multigate{4}{C_{XX}} & \qw    & \\
          \lstick{\ket{0}} & \ghost{C_{XX}}        & \qw & \qw      & \qw & \ghost{C_{XX}}        & \qw    & \\
          \vdots           & \nghost{C_{XX}}       &     & \vdots   &     & \nghost{C_{XX}}       & \vdots &\\
                           & \nghost{C_{XX}}       & \cw & b_1      &     & \nghost{C_{XX}}       & \cw    & b'_1 \\
                           & \nghost{C_{XX}}       & \cw & b_2      &     & \nghost{C_{XX}}       & \cw    & b'_2
        }
      \end{equation*}
    \end{column}
    \begin{column}{0.4\textwidth}
      \centering
      \pause
      {\color{spinsecondary}Mutual information}
      \begin{equation*}
        I(b_1 ; b_2) = 0
      \end{equation*}
      \pause
      \begin{equation*}
        I(b'_1 ; b'_2 | b_1, b_2) = 1
      \end{equation*}
    \end{column}
  \end{columns}
\end{frame}

