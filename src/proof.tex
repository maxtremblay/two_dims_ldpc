\section{Proof of the main theorem}

\begin{frame}
  \centering
  \hfill\\
  \begin{theorem}
    Let $C$ be a Clifford circuit measuring computing Pauli operators $S_1, \ldots, S_r$.
    Then, for any subset of qubits $L$, we have
    \begin{equation*}
      \text{depth}(C) 
      \geq
      \frac
        {n_{\text{cut}}}
        {64 |\partial L|}.
    \end{equation*}
  \end{theorem}
  \begin{corollary}
    For families of local-expander quantum LDPC codes of length $n$,
    a syndrome extraction circuit $C$ implemented as a local Clifford circuit 
    on a $\sqrt{N} \times \sqrt{N}$ grid of qubits
    satisfies
    \begin{equation*}
      \text{depth}(C) 
      \geq
      \Omega \left(
        \frac 
          {n}
          {\sqrt{N}}
      \right).
    \end{equation*}
  \end{corollary}
\end{frame}

\begin{frame}{Strategy}
  \hfill\\
  \begin{itemize}
    \item Partition the circuit's qubits into two subsets $L$ and $R$.
    \pause
    \item Lower bound the amount of correlation required between $L$ and $R$ 
    to measure the Pauli operators.
    \pause
    \item Upper bound the amount of correlation introduced per operation.
    \pause
    \item Combine both arguments to derive a lower bound for the depth of the circuit.
  \end{itemize}
\end{frame}

\begin{frame}{Measuring correlations}
  \hfill\\
  \Large
  \begin{columns}[c]
    \begin{column}{0.55\textwidth}
      \centering
      \begin{equation*}
        \Qcircuit @C=1em @R=1em {
          \lstick{\ket{0}} & \multigate{4}{C_{XX}} & \qw & \qw      & \qw & \multigate{4}{C_{XX}} & \qw    & \\
          \lstick{\ket{0}} & \ghost{C_{XX}}        & \qw & \qw      & \qw & \ghost{C_{XX}}        & \qw    & \\
          \vdots           & \nghost{C_{XX}}       &     & \vdots   &     & \nghost{C_{XX}}       & \vdots &\\
                           & \nghost{C_{XX}}       & \cw & b_1      &     & \nghost{C_{XX}}       & \cw    & b'_1 \\
                           & \nghost{C_{XX}}       & \cw & b_2      &     & \nghost{C_{XX}}       & \cw    & b'_2
        }
      \end{equation*}
    \end{column}
    \begin{column}{0.4\textwidth}
      \centering
      \pause
      {\color{spinsecondary}Mutual information}
      \begin{equation*}
        I(b_1 ; b_2) = 0
      \end{equation*}
      \pause
      \begin{equation*}
        I(b'_1 ; b'_2) = 1
      \end{equation*}
    \end{column}
  \end{columns}
\end{frame}

\begin{frame}{Measuring correlations}
  \hfill\\
  \Large
  \begin{columns}[c]
    \begin{column}{0.55\textwidth}
      \centering
      \begin{equation*}
        \Qcircuit @C=1em @R=1em {
          \lstick{\ket{0}} & \multigate{4}{C_{XX}} & \qw & \gate{E_1} & \qw & \multigate{4}{C_{XX}} & \qw    & \\
          \lstick{\ket{0}} & \ghost{C_{XX}}        & \qw & \gate{E_2} & \qw & \ghost{C_{XX}}        & \qw    & \\
          \vdots           & \nghost{C_{XX}}       &     & \vdots     &     & \nghost{C_{XX}}       & \vdots & \\
                           & \nghost{C_{XX}}       & \cw & b_1        &     & \nghost{C_{XX}}       & \cw    & b'_1 \\
                           & \nghost{C_{XX}}       & \cw & b_2        &     & \nghost{C_{XX}}       & \cw    & b'_2
        }
      \end{equation*}
    \end{column}
    \begin{column}{0.4\textwidth}
      \centering
      {\color{spinsecondary}Mutual information}
      \begin{equation*}
        I(b_1', b_2', E_1 ; E_2) = 1
      \end{equation*}
    \end{column}
  \end{columns}
\end{frame}

\begin{frame}[c]{Measuring correlations}
  \hfill\\
  \Large
  \begin{columns}[c]
    \begin{column}{0.4\textwidth}
      \centering
      \begin{equation*}
        \Qcircuit @C=1em @R=1em {
          \lstick{\ket{+}} & \gate{M_Z} \cwx[1] & \qw & & \ket{00} \\
          \lstick{\ket{0}} & \gate{X}           & \qw & & \ket{11} 
          \gategroup{1}{3}{2}{3}{1em}{\}} 
        }
      \end{equation*}
    \end{column}
    \pause
    \begin{column}{0.4\textwidth}
      \centering
      Classical operations can artificially boost mutual information.
    \end{column}
  \end{columns}
\end{frame}

\begin{frame}[c]{Measuring correlations}
  \hfill\\
  \Large
  \begin{itemize}
    \item
      Build a circuit $C'$ with the same action and similar depth as $C$ by pushing
      all measurements and classical operations at the end.
    \pause
    \item
      Consider the circuit $C' \circ E \circ C'$.
    \pause
    \item
      Compute the mutual information
      \begin{equation*}
        I(O^{(2)}_L, E_L; O^{(2)}_R, E_R | O^{(1)}).
      \end{equation*}
  \end{itemize}
\end{frame}


\begin{frame}{Circuit transformations}
  \Large
  \begin{columns}[c]
    \begin{column}{0.4\textwidth}
      \centering
      \begin{equation*}
        \Qcircuit @C=1em @R=1em {
          \lstick{}  & \multigate{2}{M_{PQ}} & \qw \\
                     & \cghost{M_{PQ}}       & \cw 
          \inputgroupv{1}{2}{1em}{1em}{L}\\
          \lstick{R} & \ghost{M_{PQ}}        & \qw   
        }
      \end{equation*}
    \end{column}
    \begin{column}{0.4\textwidth}
      \centering
      \begin{equation*}
        \Qcircuit @C=1em @R=1em {
          \lstick{}   & \qw & \qw              & \gate{P}  & \qw      & \qw        & \qw \\
          \lstick{}   &     & \lstick{\ket{+}} & \ctrl{-1} & \ctrl{2} & \gate{M_X}     \\
                      & \cw & \cw              & \cw       & \cw      & \cctrlo{-1}        & \cw 
          \inputgroupv{1}{3}{1em}{2em}{L'}\\
           \lstick{R'} & \qw &\qw               & \qw       & \gate{Q} & \qw        & \qw  
        }
      \end{equation*}
    \end{column}
  \end{columns}
  \hfill\\
  \centering
  \begin{equation*}
    \text{depth}(C') \leq 4 \cdot \text{depth}(C) + 2
  \end{equation*}
\end{frame}

\begin{frame}[c]{Circuit transformations}
  \Large
  \begin{columns}[c]
    \begin{column}{0.5\textwidth}
      \centering
      \begin{equation*}
        \Qcircuit @C=0.5em @R=0.5em {
          \lstick{}          & \multigate{2}{U_1} & \qw        & \qw & \qw & \qw & \qw & \qw                & \multigate{2}{U_2} & \qw \\
          \lstick{}          & \ghost{U_1}        & \qw        & \qw & \qw & \qw & \qw & \qw                & \ghost{U_2}        & \qw \\
          \lstick{\ket{s_1}} & \ghost{U_1}        & \gate{M_1} &     &     &     &     & \lstick{\ket{s_2}} & \ghost{U_2}        & \gate{M_2} \\ 
        }
      \end{equation*}
    \end{column}
    \begin{column}{0.4\textwidth}
      \centering
      \begin{equation*}
        \Qcircuit @C=0.5em @R=0.5em {
          \lstick{\ket{s_1}} & \multigate{2}{U_1} & \qw & \qw                & \gate{M_1} \\ 
          \lstick{}          & \ghost{U_1}        & \qw & \multigate{2}{U_2} & \qw \\
          \lstick{}          & \ghost{U_1}        & \qw & \ghost{U_2}        & \qw \\
          \lstick{\ket{s_2}} & \qw                & \qw & \ghost{U_2}        & \gate{M_2} \\ 
        }
      \end{equation*}
    \end{column}
  \end{columns}
  \hfill\\[10mm]
  \centering
  \pause
  \textbf{\color{spinsecondary}
    Both ancillas are the same node in the connectivity graph 
    and in the same partition.
  }
\end{frame}

\begin{frame}[c]{Circuit transformations}
  \Large
  \begin{columns}[c]
    \begin{column}{0.4\textwidth}
      \centering
      \only<1-3>{
        \begin{equation*}
          \Qcircuit @C=0.5em @R=0.5em {
            & \gate{P} & \gate{G} & \qw \\
            & \cctrl{-1} & \cw & \cw
          }
        \end{equation*}
      }
      \hfill \\
      \only<2-3>{
        \begin{equation*}
          \Qcircuit @C=0.5em @R=0.5em {
            & \gate{P}     & \gate{M}    & \qw \\
            & \cctrl{-1}   & \cw         & \cw \\
            & \cw          & \cctrlo{-2} & \cw
          }
        \end{equation*}
      }
      \hfill \\
      \only<3>{
        \begin{equation*}
          \Qcircuit @C=0.5em @R=0.5em {
            & \gate{P_1} & \gate{P_2} & \qw & & \cdots & & &  \gate{P_s} & \qw \\
            & \cctrl{-1} & \cctrl{-1} & \qw & & \cdots & & &  \cctrl{-1} & \cw
          }
        \end{equation*}
      }
    \end{column}
    \begin{column}{0.4\textwidth}
      \centering
      \only<1-3>{
        \begin{equation*}
          \Qcircuit @C=0.5em @R=0.5em {
            & \gate{G} & \gate{Q} & \qw \\
            & \cw      & \cctrl{-1} &  \cw
          }
        \end{equation*}
      }
      \hfill \\
      \only<2-3>{
        \begin{equation*}
          \Qcircuit @C=0.5em @R=0.5em {
            & \gate{M}    & \gate{P}   & \qw & \qw \\
            & \cw         & \cctrl{-1} & \cw & \cw \\
            & \cctrlo{-2} & \cw        & \cw & \cw
          }
        \end{equation*}
      }
      \hfill \\
      \only<3> {
      \begin{equation*}
          \Qcircuit @C=0.5em @R=0.5em {
            & \gate{P} & \qw \\
            & \cctrl{-1} & \cw 
          }
        \end{equation*}
      }
    \end{column}
  \end{columns}
\end{frame}

\begin{frame}[c]{Circuit transformations}
  \centering
  \Large
  \begin{columns}[c]
    \begin{column}{0.4\textwidth}
      \centering
      \begin{equation*}
        \Qcircuit @C=1.5em @R=1.5em {
          \lstick{\boldsymbol{D}}  & \multigate{1}{U} & \qw       & \gate{P} & \qw  \\
          \lstick{\boldsymbol{A'}} & \ghost{U}        & \gate{M}  & \cctrl{-1} &\\
        }
      \end{equation*}
    \end{column}
    \begin{column}{0.4\textwidth}
      \centering
      \pause
      \begin{equation*}
        \text{depth}(U) \leq 4 \cdot \text{depth}(C) 
      \end{equation*}
      \hfill \\
      \pause
      \begin{equation*}
        |\partial L| = |\partial L'|
      \end{equation*}
    \end{column}
  \end{columns}
\end{frame}

\begin{frame}[c]{The double measurement circuit}
  \centering
  \Large
  \begin{equation*}
    \Qcircuit @C=1em @R=1em {
      \lstick{\boldsymbol{A_E}}  & \ket{+} &     & \gate{M_E}        & \cctrl{1} \\
      \lstick{\boldsymbol{D}}    & \qw     & \qw & \multigate{1}{U}  & \gate{E} & \multigate{2}{U} & \qw & \qw & \qw        & \gate{P_1}         & \qw        & \gate{P_2} & \qw \\
      \lstick{\boldsymbol{A'_1}} & \qw     & \qw &\ghost{U}         & \qw      & \nghost{U}       &     & \qw & \gate{M_1} & \cctrl{1} \cwx{-1} &            & \\
      \lstick{\boldsymbol{A'_2}} & \qw     & \qw & \qw               & \qw      & \ghost{U}        & \qw & \qw & \qw        & \gate{P'_1}        & \gate{M_2} & \cctrl{-2} \\
    }
  \end{equation*}
  \hfill\\
  $
    I(O_{\bar L}^{(2)}, E_{\bar L} ; O_{\bar R}^{(2)}, E_{\bar R} | O^{(1)})
  $
  with $\bar L = L' \cup A_E$
\end{frame}

\begin{frame}[c]{Bounds on the mutual information}
  \hfill\\
  \Large
  \textbf{\color{spinsecondary} Lower bound} \\
  For the double measurement circuit, we have
  \begin{equation*}
    I(O_{\bar L}^{(2)}, E_{\bar L} ; O_{\bar R}^{(2)}, E_{\bar R} | O^{(1)})
    \geq \frac{n_{\text{cut}}}{2}.
  \end{equation*}
  \\
  \pause
  \textbf{\color{spinsecondary} Upper bound} \\
  For the double measurement circuit, we have
  \begin{equation*}
    I(O_{\bar L}^{(2)}, E_{\bar L} ; O_{\bar R}^{(2)}, E_{\bar R} | O^{(1)})
    \leq 32|\partial L| \text{depth}(C).
  \end{equation*}
\end{frame}

\begin{frame}[c]{Proof of the lower bound}
  \hfill\\
  Note $m_i^{(t)} = \bigoplus_{o \in O_i^{(t)}} o$, the outcome
  for the measurement of $S_i$ in circuit $t$.
  \\[3mm]
  \pause
  Note $M_{\bar L}^{(t)} = \lbrace m_{i, \bar L}^{(t)} = \bigoplus_{o \in O_i^{(t)} \cap \bar L} o \rbrace$ and 
  similarly for $M_{\bar R}^{(t)}$.
  \\[3mm]
  \pause 
  From the data processing inequality
  \begin{align*}
    I(O_{\bar L}^{(2)}, E_{\bar L} ; O_{\bar R}^{(2)}, E_{\bar R} | O^{(1)})
    &\geq 
    I(M_{\bar L}^{(2)}, E_{\bar L} ; M_{\bar R}^{(2)}, E_{\bar R} | O^{(1)}) \\
    \onslide<4->{
      &= H(M_{\bar L}^{(2)}, E_{\bar L} | O^{(1)})
      - H(M_{\bar L}^{(2)}, E_{\bar L} | M_{\bar R}^{(2)}, E_{\bar R}, O^{(1)}) \\
    }
    \onslide<5->{
      &= H(M_{\bar L}^{(2)}, E_{\bar L}, O^{(1)}) - H(O^{(1)})
      - H(E_{\bar L}) \\
    }
    \onslide<6->{
      &= H(M_{\bar L}^{(2)}| E_{\bar L}, O^{(1)}).
    }
  \end{align*}
\end{frame}

\begin{frame}[c]{Proof of the lower bound}
  \hfill\\
  Note $S_{\text{cut}}$ the operators with support on both $L$ and $R$.
  \\[3mm]
  \pause
  Note $S_{\text{cut}, \bar L}$ the operators $S_i \in S_{\text{cut}}$ for which
  $m_i$ depends on at least one outcome in $O_L$.
  \\[3mm]
  \pause
  Note $M_{\bar L,\text{cut}}^{(t)}$ the outcome of $M_{\bar L}^{(t)}$ corresponding to $S_{\text{cut}, \bar L}$.
  \\[3mm]
  \pause
  From the data processing inequality
  \begin{align*}
    H(M_{\bar L}^{(2)}| E_{\bar L}, O^{(1)})
    &\geq 
    H(M_{\bar L, \text{cut}}^{(2)}| E_{\bar L}, O^{(1)}) \\
    \onslide<5->{
      &\geq 
      H(M_{\bar L, \text{cut}}^{(2)}| E_{\bar L}, O^{(1)}, M_{\bar R}^{(2)}) \\
    }
    \onslide<6->{
      &= 
      H(m_i(E_{\bar R})\, : \, S_i \in S_{\text{cut}, \bar L} | E_{\bar L}, O^{(1)},  M_{\bar R}^{(2)}) \\
    }
    \onslide<7->{
      &= 
      H(m_i(E_{\bar R})\, : \, S_i \in S_{\text{cut}, \bar L}) \\
    }
    \onslide<8->{
      &= 
      |S_{\text{cut}, \bar L}|.
    }
  \end{align*}
\end{frame}

\begin{frame}[c]{Proof of the lower bound}
  \hfill\\
  \Large
  By symmetry 
  \begin{equation*}
    I(O_{\bar L}^{(2)}, E_{\bar L} ; O_{\bar R}^{(2)}, E_{\bar R} | O^{(1)})
    \geq 
    \max \lbrace 
      |S_{\text{cut}, \bar L}|, 
      |S_{\text{cut}, \bar R}| 
    \rbrace
    \geq \frac{n_{\text{cut}}}{2}.
  \end{equation*}
\end{frame}

\begin{frame}[c]{Proof of the upper bound}
  \centering
  \large
  \begin{equation*}
    \Qcircuit @C=1em @R=1em {
      \lstick{\boldsymbol{A_E}}  & \ket{+} \barrier[0.5em]{4} &     & \gate{M_E} \barrier[-0.9em]{4} & \cctrl{1} \barrier[-1.1em]{4} & \barrier[-0.5em]{4} &     &            & \barrier[-1.2em]{4}  & \barrier[-1.2em]{4} &            & \\
      \lstick{\boldsymbol{D}}    & \qw                        & \qw & \multigate{1}{U}               & \gate{E}                      & \multigate{2}{U}    & \qw & \qw        & \gate{P_1}           & \qw                 & \gate{P_2} & \qw \\
      \lstick{\boldsymbol{A'_1}} & \qw                        & \qw & \ghost{U}                      & \qw                           & \nghost{U}          &     & \gate{M_1} & \cctrl{1} \cwx{-1}   &                     &            & \\
      \lstick{\boldsymbol{A'_2}} & \qw                        & \qw & \qw                            & \qw                           & \ghost{U}           & \qw & \qw        & \gate{P'_1}          & \gate{M_2}          & \cctrl{-2} & \\
                                 & \rstick{t_0}               &     & \rstick{t_1}                   & \rstick{t_2}                  & \rstick{t_3}        &     &            & \rstick{t_4}         & \rstick{t_5}        &            &
    }
  \end{equation*}
  \hfill \\
  \begin{equation*}
    S_{A_2', A_E}(\rho_{\bar L}(t_5); \rho_{\bar R}(t_5))
    =
    I(O_{\bar L}^{(2)}, E_{\bar L}; O_{\bar R}^{(2)}, E_{\bar R})
    \geq
    I(O_{\bar L}^{(2)}, E_{\bar L}; O_{\bar R}^{(2)}, E_{\bar R} | O^{(1)})
  \end{equation*}
\end{frame}

\begin{frame}[c]{Proof of the upper bound}
  \hfill \\[1cm]
  \begin{itemize}
    \large
    \item  
    Given a set of qubits and a partition into subsets $A$, $B$.
    Let $\rho$ be a density matrix on $A \cup B$ and 
    $G$ be a two-qubit unitary gate acting qubit of $A$ and a qubit of $B$.
    Note $\rho' = G \rho G^\dag$, then
    \begin{equation*}
      S(\rho'_A ; \rho'_B) \leq S(\rho_A, \rho_B) + 4.
    \end{equation*}
    \only<1-> {
    \item
      Single qubit gates and measurements are CPTP maps. 
      Thus, they can't increase the mutual entropy.
    }
    \only<2-> {
    \item
      Discarding a subsystem can't increase the mutual entropy.
    }
  \end{itemize}

  \hfill\\[1.5cm]
  {\large \color{spingrey}
    Mariën et al, Communications in Mathematical Physics, 346(1), 2016
  }
\end{frame}


\begin{frame}[c]{Proof of the upper bound}
  \hfill \\[1cm]
  \Large
\end{frame}

\begin{frame}[c]{Proof of the upper bound}
  \centering
  \large
  \begin{equation*}
    \Qcircuit @C=1em @R=1em {
      \lstick{\boldsymbol{A_E}}  & \ket{+} \barrier[0.5em]{4} &     & \gate{M_E} \barrier[-0.9em]{4} & \cctrl{1} \barrier[-1.1em]{4} & \barrier[-0.5em]{4} &     &            & \barrier[-1.2em]{4}  & \barrier[-1.2em]{4} &            & \\
      \lstick{\boldsymbol{D}}    & \qw                        & \qw & \multigate{1}{U}               & \gate{E}                      & \multigate{2}{U}    & \qw & \qw        & \gate{P_1}           & \qw                 & \gate{P_2} & \qw \\
      \lstick{\boldsymbol{A'_1}} & \qw                        & \qw & \ghost{U}                      & \qw                           & \nghost{U}          &     & \gate{M_1} & \cctrl{1} \cwx{-1}   &                     &            & \\
      \lstick{\boldsymbol{A'_2}} & \qw                        & \qw & \qw                            & \qw                           & \ghost{U}           & \qw & \qw        & \gate{P'_1}          & \gate{M_2}          & \cctrl{-2} & \\
                                 & \rstick{t_0}               &     & \rstick{t_1}                   & \rstick{t_2}                  & \rstick{t_3}        &     &            & \rstick{t_4}         & \rstick{t_5}        &            &
    }
  \end{equation*}
  \hfill \\
  \only<1> {
    \begin{equation*}
      S(\rho_{\bar L}(t_0); \rho_{\bar R}(t_0))
      = 0
    \end{equation*}
  }
  \only<2> {
    \begin{equation*}
      S(\rho_{\bar L}(t_1); \rho_{\bar R}(t_1))
      \leq 4 \text{depth}(U) |\partial L|
    \end{equation*}
  }
  \only<3> {
    \begin{equation*}
      S(\rho_{\bar L}(t_2); \rho_{\bar R}(t_2))
      \leq 4 \text{depth}(U) |\partial L|
    \end{equation*}
  }
  \only<4> {
    \begin{equation*}
      S(\rho_{\bar L}(t_3); \rho_{\bar R}(t_3))
      \leq 8 \text{depth}(U) |\partial L|
    \end{equation*}
  }
  \only<5> {
    \begin{equation*}
      S(\rho_{\bar L}(t_4); \rho_{\bar R}(t_4))
      \leq 8 \text{depth}(U) |\partial L|
    \end{equation*}
  }
  \only<6> {
    \begin{equation*}
      S_{A_2', A_E}(\rho_{\bar L}(t_4); \rho_{\bar R}(t_4))
      \leq 8 \text{depth}(U) |\partial L|
    \end{equation*}
  }
  \only<7> {
    \begin{equation*}
      S_{A_2', A_E}(\rho_{\bar L}(t_5); \rho_{\bar R}(t_5))
      \leq 8 \text{depth}(U) |\partial L|
    \end{equation*}
  }
\end{frame}

\begin{frame}[c]{Main theorem}
  \Large
  \begin{align*}
    \frac{n_{\text{cut}}}{2}
    &\leq
    I(O_{\bar L}^{(2)}, E_{\bar L}; O_{\bar R}^{(2)}, E_{\bar R} | O^{(1)}) \\
    &\leq 8 \text{depth}(U) |\partial L| \\
    &\leq 32 \text{depth}(C) |\partial L|
  \end{align*}
\end{frame}

\begin{frame}[c]{Corollary}
  \Large
  For families of local-expander quantum LDPC codes of length $n$,
  a syndrome extraction circuit $C$ implemented as a local Clifford circuit 
  on a $\sqrt{N} \times \sqrt{N}$ grid of qubits
  satisfies
  \begin{equation*}
    \text{depth}(C) 
    \geq
    \Omega \left(
      \frac 
        {n}
        {\sqrt{N}}
    \right).
  \end{equation*}
\end{frame}

\begin{frame}[c]{Proof}
  \centering
  \Large
  \begin{columns}[c]
    \begin{column}{0.4\textwidth}
      \centering
      \begin{tikzpicture}
        \node (L) at (2.5, 5.5) {$\sqrt{N}$};
        \node (L) at (5.5, 2.5) {$\sqrt{N}$};
        \fill[spingrey!20] (0,0) rectangle (5, 5);

        \node (L) at (0.5, 4.5) {$L$};
        \node (R) at (4.5, 4.5) {$R$};

        \fill[spinprimarylighter] (1.35,2) circle (1.1);
        \fill[spinprimary] (1.35, 2) circle (0.2);

        \draw[line width = 0.15em] (1.75, -0.25) -- (1.75, 5.25);
      \end{tikzpicture}
    \end{column}
    \begin{column}{0.6\textwidth}
      \begin{itemize}
        \item
          For all $\varepsilon \in [0, 1]$,
          we can move the line such that  \\
          $
            \varepsilon n/2 - \sqrt{N} \leq |D \cap L| \leq \varepsilon n / 2.
          $
        \pause
        \item
          In the connectivity graph
          $|\partial L| \leq a \sqrt{N}$
        \pause
        \item
          In the contracted Tanner graph
          $n_{\text{cut}} \geq b h_{\varepsilon}(\bar T) |D \cap L| $
        \pause
        \item
          $\text{depth}(C) \geq c \varepsilon h_{\varepsilon}(\bar T)n / \sqrt{N}$
      \end{itemize}
    \end{column}
  \end{columns}
\end{frame}
