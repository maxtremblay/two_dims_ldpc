\section{Background and definitions}

% Clifford circuits

\begin{frame}{Clifford circuit}
  \centering
  \begin{itemize}
    \item Preparations of \ket{0} and \ket{+} and classical bits.
    \pause
    \item Single-qubit and two-qubit Pauli measurements.
    \pause
    \item Single-qubit and two-qubit unitary Clifford gates.
    \pause
    \item Classically-controlled Pauli operators.
    \pause
    \item Ouput the parity of some subsets of measurement outcomes.
  \end{itemize}
\end{frame}

\begin{frame}{Clifford circuit}
  \centering
  \begin{equation*}
  \Qcircuit @C=1.25em @R=1.25em { 
    \lstick{\ket{\psi_1}} & \qw                   & \qw       & \qw                   & \qw              & \qw & \ctrl{1}  & \qw & \qw \\ 
    \lstick{\ket{0}}      & \multigate{2}{M_{XX}} & \qw       & \qw                   & \qw              & \qw & \targ     & \qw & \gate{M_Z} \\ 
    \lstick{0}            & \cghost{M_{XX}}       & \cctrl{1} &                       &                  &     &           &     &     \\ 
    \lstick{\ket{0}}      & \ghost{M_{XX}}        & \gate{Z}  & \multigate{2}{M_{XX}} & \qw              & \qw & \qw       & \qw & \gate{M_Z} \\ 
    \lstick{0}            & \cw                   & \cw       & \cghost{M_{XX}}       & \cctrl{1}        &     &           &     &      &  & \text{parity} \\ 
    \lstick{\ket{0}}      & \multigate{2}{M_{XX}} & \qw       & \ghost{M_{XX}}        & \gate{Z} \cwx[2] & \qw & \qw       & \qw & \gate{M_Z} \\ 
    \lstick{0}            & \cghost{M_{XX}}       & \cctrl{1} &                       &                  &     &           &     &     \\ 
    \lstick{\ket{0}}      & \ghost{M_{XX}}        & \gate{Z}  & \qw                   & \gate{Z}         & \qw & \targ     & \qw & \gate{M_Z} \\ 
    \lstick{\ket{\psi_2}} & \qw                   & \qw       & \qw                   & \qw              & \qw & \ctrl{-1} & \qw & \qw 
    \gategroup{2}{8}{8}{9}{1em}{\}}
  }
  \end{equation*}
\end{frame}

\begin{frame}{Local circuit}
  \centering
  \Large
  \hfill\\[3mm]
  A $b$-local circuit is a circuit with qubits placed on a subset of the $\mathbb Z \times \mathbb Z$
  grid such that any two-qubit operation acts on qubits at distance at most $b$ from each other.
  \\[6mm]
  \pause
  \begin{tikzpicture}
    \node
      [diamond, fill = spinsecondarylighter, scale=3.8] 
      (d) at (1,2) {};

    \foreach \x in {0,...,7}
      \foreach \y in {0,...,7} 
        \fill[spinprimary] (0.5 * \x, 0.5 * \y) circle (0.1);
    \fill[spinsecondary] (1.0, 2.0) circle (0.1);
  \end{tikzpicture}
\end{frame}

\begin{frame}[c]{Connectivity graph}
  \centering
  \begin{columns}[c]
    \begin{column}{0.4\textwidth}
      \centering
      \begin{equation*}
        \Qcircuit @C=1.25em @R=1.25em { 
          \lstick{1} & \ctrl{1} & \gate{H}  & \targ     & \qw \\
          \lstick{2} & \targ    & \targ     & \ctrl{-1} & \qw \\
          \lstick{3} & \gate{H} & \qw       & \ctrl{1}  & \qw \\
          \lstick{4} & \ctrl{2} & \qw       & \targ     & \qw \\ 
          \lstick{5} & \qw      & \ctrl{-3} & \targ     & \qw \\
          \lstick{6} & \targ    & \gate{H}  & \ctrl{-1} & \qw 
        }
      \end{equation*}
    \end{column}
    \pause
    \begin{column}{0.4\textwidth}
      \centering
      \begin{tikzpicture}
        \draw[line width=0.5mm] (0, 0) -- (0,1);
        \draw[line width=0.5mm] (0, 2) -- (0,3);
        \draw[line width=0.5mm] (0, 4) -- (0,5);
        \draw[line width=0.5mm] (0,1) arc [
          start angle=-90,
          end angle=90,
          x radius=0.75,
          y radius=1.5
        ];
        \draw[line width=0.5mm] (0,2) arc [
          start angle=90,
          end angle=270,
          x radius=0.75,
          y radius=1
        ];

        \foreach \y in {0,...,5} 
          \fill[spinprimary] (0, \y) circle (0.2);

      \end{tikzpicture}
    \end{column}
  \end{columns}
\end{frame}

\begin{frame}[c]{Stabilizer code}

  \centering
  \stitle{Stabilizer group} 
  \\[3mm]
  $
    \mathcal S 
    = \langle S_1, S_2, \ldots, S_r \rangle
    \subset \mathcal P_n \setminus \lbrace -I \rbrace
  $
  such that $S_i S_j = S_j S_i$

  \hfill\\
  \pause

  \stitle{Stabilizer code}
  \\[3mm]
  common $+1$ eigenspace of $\mathcal S$
  \begin{equation*}
    C(S) = 
    \lbrace 
      \ket \psi 
      \, : \, 
      S \ket \psi = \psi 
      \, \forall S \in \mathcal S 
    \rbrace
  \end{equation*}

  \pause

  \stitle{Example}
  \\[3mm]
  The five qubits code
  \begin{equation*}
    \mathcal S =
    \langle
      XXZIZ,
      ZXXZI,
      IZXXZ,
      ZIZXX
    \rangle
  \end{equation*}

\end{frame}

\begin{frame}[c]{Pauli measurement circuit}
  \Large
  \begin{itemize}
    \item  
    Consider $n$-qubit independent commuting Pauli operators
    $S_1, \ldots, S_r$.
    \pause
    \item
    For $\boldsymbol{m} = (m_1, \ldots, m_r) \in \mathbb Z_2^r$,
    denote $\Pi_{\boldsymbol{m}}$ the projector onto the common eigenspace of $S_i$ 
    with value $(-1)^{m_i}$.
    \pause
    \item
    A Pauli measurement circuit maps a $n$-qubit state $\rho$ to
    $\Pi_{\boldsymbol{m}} \rho \Pi_{\boldsymbol{m}}$ with probability $\text{Tr}(\Pi_{\boldsymbol{m}}\rho)$
    and output $\boldsymbol{m}$.
    \pause
    \item
    The circuit use $N = n + a$ qubits where $a$ is the number of ancilla qubits.
  \end{itemize}
\end{frame}

\begin{frame}[c]{Tanner graph}
  \centering
  \Large
  \begin{equation*}
    S = 
    \lbrace
     XXIII,
     ZXIZI,
     IIYZZ
    \rbrace
  \end{equation*}
  \begin{columns}[c]
    \pause
    \begin{column}{0.45\textwidth}
      \centering
      $T(S) = ({\color{spinprimary}V_Q} \cup {\color{spinsecondary}S}, E)$
      \\
      $\lbrace q_i, S_j\rbrace \in E$ iff $S_j$ acts non-trivially on $q_i$
    \end{column}
    \pause
    \begin{column}{0.4\textwidth}
      \centering
      \begin{tikzpicture}
        \draw[line width = 0.5mm] (0, 4) -- (3, 3.5);
        \draw[line width = 0.5mm] (0, 3) -- (3, 3.5);

        \draw[line width = 0.5mm] (0, 4) -- (3, 2);
        \draw[line width = 0.5mm] (0, 3) -- (3, 2);
        \draw[line width = 0.5mm] (0, 1) -- (3, 2);

        \draw[line width = 0.5mm] (0, 2) -- (3, 0.5);
        \draw[line width = 0.5mm] (0, 1) -- (3, 0.5);
        \draw[line width = 0.5mm] (0, 0) -- (3, 0.5);

        \foreach \y in {0, ..., 4}
          \fill[spinprimary] (0, \y) circle (0.2);
        \foreach \y in {0, ..., 2}
          \fill[spinsecondary] (2.8, 1.5*\y+0.3) rectangle (3.2, 1.5*\y+0.7);
      \end{tikzpicture}
    \end{column}
  \end{columns}
\end{frame}

\begin{frame}[c]{Contracted Tanner graph}
  \centering
  \Large
  \begin{equation*}
    S = 
    \lbrace
     XXIII,
     ZXIZI,
     IIYZZ
    \rbrace
  \end{equation*}
  \begin{columns}[c]
    \pause
    \begin{column}{0.45\textwidth}
      \centering
      $\bar{T}(S) = ({\color{spinprimary}V_Q}, \bar{E})$
      \\
      $\lbrace q_i, q_j\rbrace \in \bar E$ iff $\exists S_k$ actings non-trivially on $q_i$
    \end{column}
    \pause
    \begin{column}{0.4\textwidth}
      \centering
      \begin{tikzpicture}
        \draw[line width = 0.5mm] (0, 4) -- (0, 3);
        \draw[line width=0.5mm] (0,4) arc [
          start angle=90,
          end angle=270,
          x radius=0.75,
          y radius=1.5
        ];

        \draw[line width=0.5mm] (0,3) arc [
          start angle=90,
          end angle=-90,
          x radius=0.75,
          y radius=1
        ];

        \draw[line width = 0.5mm] (0, 2) -- (0, 1);
        \draw[line width=0.5mm] (0,2) arc [
          start angle=90,
          end angle=270,
          x radius=0.75,
          y radius=1
        ];

        \draw[line width = 0.5mm] (0, 1) -- (0, 0);

        \foreach \y in {0, ..., 4}
          \fill[spinprimary] (0, \y) circle (0.2);
      \end{tikzpicture}
    \end{column}
  \end{columns}
\end{frame}


\begin{frame}[c]{Quantum LDPC codes}
  \centering
  \Large
  A family of quantum LDPC codes $(Q_i)_i$ is a family
  of stabilizer codes such that the degree of the Tanner graph $T_i$ 
  are bounded by some constant independent of $i$.
\end{frame}

\begin{frame}[c]{Local-expansion}
  \centering
  \Large
  The Cheeger constant of a graph $G = (V, E)$ is defined as 
  \begin{equation*}
    h_{\varepsilon}(G) 
    = 
    \min_{\substack{
      L \subseteq V \\
      |L| \leq \varepsilon |V| / 2
    }}
    \frac
      {|\partial L|}
      {|L|}.
  \end{equation*}
  \\[2mm]
  \pause
  A family of graphs $(G_i)_{i\in \mathbb N}$ is an $(\alpha, \varepsilon)$-expander
  graph family if $h_{\varepsilon}(G_i) \geq \alpha$ for all $i$.
  \\[4mm]
  \pause
  A family of quantum local-expander codes is a family of stabilizer codes with $(\alpha, \varepsilon)$-expander
  contracted Tanner graphs.
\end{frame}

\begin{frame}[c]{Lemma}
  \Large
  \hfill\\
  Let $T$ be the Tanner graph of a stabilizer code 
  with length $n$ and with $r$ stabilizer generators.
  Let $\bar T$ be its contracted Tanner graph.
  Then, for all $\varepsilon \in [0, 1]$, we have
  \begin{equation*}
    h_{\varepsilon'}(\bar T) 
    \geq 
    \frac
      {h_\varepsilon(T)}
      {\text{deg}(T)}
  \end{equation*}
  where
  \begin{equation*}
    \varepsilon' =
    \left( 1 + \frac{r}{n} \right)
    \frac{\varepsilon}{1 + \text{deg}(T)}.
  \end{equation*}
\end{frame}

\begin{frame}
  \Large
  \begin{itemize}
    \item Show that 
      $$
        \left| \partial_{\bar T} L \right|
        \geq 
        \frac
        { \left| \partial_{T}(L \cup N_T(L)) \right| }
        { \text{deg}(T) }.
      $$
    \pause
    \item If 
      $$
         \left| L \cup N_T(L) \right|
         \leq 
         \varepsilon \frac{V(T)}{2},
      $$ 
      then
      $$
         \left| \partial_{T}(L \cup N_T(L)) \right|
         \geq 
         h_{\varepsilon}(T)
         \left| L \cup N_T(L) \right|
         \geq 
         h_{\varepsilon}(T)
         \left| L \right|.
      $$
    \pause
    \item Combine the two
      $$
        \frac
          { \left| \partial_{\bar T} L \right| }
          { \left| L \right| }
        \geq 
        \frac
        {h_{\varepsilon}(T)}
        {\text{deg}(T)}.
      $$
  \end{itemize}
\end{frame}

\begin{frame}
  \Large
  \centering
  $$
    \left| \partial_{\bar T} L \right|
    \text{deg}(T)
    \geq 
    \left| \partial_{T}(L \cup N_T(L)) \right|
  $$
  \begin{columns}[c]
    \begin{column}{0.45\textwidth}
      \centering
      \begin{tikzpicture}
        \draw[line width = 0.5mm] (0,0) rectangle (6,4);
        \node (V) at (0.7,0.4) {$V(\bar T)$};

        \fill[spinprimarylighter!50] (1.75,2.25) circle (1.25);
        \node[text=spinprimary] (L) at (1.1,2.8) {$L$};

        \draw[line width = 0.5mm] (2.25, 1.75) -- (5, 2);

        \fill[spinprimary] (2.25,1.75) circle (0.15);

        \fill[spinprimary] (5,2) circle (0.15);
      \end{tikzpicture}
    \end{column}

    \begin{column}{0.45\textwidth}
      \centering
      \begin{tikzpicture}
        \draw[line width = 0.5mm] (0,0) rectangle (6,4);
        \node (V) at (0.7,0.4) {$V(T)$};

        \node
          [ellipse, fill = spinsecondarylighter!50, scale=7.4] 
          (e) at (2.3,2.2) {\,};
        \node[text=spinsecondary] (N) at (3.6,2.8) {$N_T(L)$};

        \fill[spinprimarylighter!50] (1.75,2.25) circle (1.25);
        \node[text=spinprimary] (L) at (1.1,2.8) {$L$};

        \draw[line width = 0.5mm] (2.25, 1.75) -- (3.7, 2.0);
        \draw[line width = 0.5mm] (2.25, 1.75) -- (3.2, 1.2);
        \draw[line width = 0.5mm] (3.7, 2.0) -- (5, 2);
        \draw[line width = 0.5mm] (3.2, 1.2) -- (5, 2);

        \fill[spinprimary] (2.25,1.75) circle (0.15);
        \fill[spinprimary] (5,2) circle (0.15);

        \fill[spinsecondary] (3.5, 1.8) rectangle (3.9, 2.2);
        \fill[spinsecondary] (3.0, 1.0) rectangle (3.4, 1.4);
      \end{tikzpicture}
    \end{column}
  \end{columns}
\end{frame}

\begin{frame}[c]{Review}
  \begin{itemize}
    \item (Local) Clifford circuits
    \item Connectivity graphs
    \item Stabilizer codes and LDPC codes
    \item Pauli measurement circuits
    \item (Contracted) Tanner graphs
    \item Expander codes
  \end{itemize}
\end{frame}
